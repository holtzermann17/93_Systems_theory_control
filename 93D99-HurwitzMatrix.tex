\documentclass[12pt]{article}
\usepackage{pmmeta}
\pmcanonicalname{HurwitzMatrix}
\pmcreated{2013-03-22 14:02:45}
\pmmodified{2013-03-22 14:02:45}
\pmowner{lha}{3057}
\pmmodifier{lha}{3057}
\pmtitle{Hurwitz matrix}
\pmrecord{4}{35395}
\pmprivacy{1}
\pmauthor{lha}{3057}
\pmtype{Definition}
\pmcomment{trigger rebuild}
\pmclassification{msc}{93D99}
\pmdefines{Hurwitz transfer function}
\pmdefines{stability matrix}

% this is the default PlanetMath preamble.  as your knowledge
% of TeX increases, you will probably want to edit this, but
% it should be fine as is for beginners.

% almost certainly you want these
\usepackage{amssymb}
\usepackage{amsmath}
\usepackage{amsfonts}

% used for TeXing text within eps files
%\usepackage{psfrag}
% need this for including graphics (\includegraphics)
%\usepackage{graphicx}
% for neatly defining theorems and propositions
%\usepackage{amsthm}
% making logically defined graphics
%%%\usepackage{xypic}

% there are many more packages, add them here as you need them

% define commands here
\begin{document}
A square matrix $A$ is called a {\em Hurwitz matrix} if all eigenvalues of $A$ have strictly negative real part, $Re[\lambda_i] < 0$; $A$ is also called a {\em stability matrix}, because the feedback system
$$
    \dot x = A x
$$
is stable.

If $G(s)$ is a (matrix-valued) transfer function, then $G$ is called Hurwitz if the poles of all elements of $G$  have negative real part.  Note that it is not necessary that $G(s)$, for a specific argument $s$, be a Hurwitz matrix --- it need not even be square.  The c{}onnection is that if $A$ is a Hurwitz matrix, then the dynamical system
\begin{eqnarray*}
  \dot x(t) &=& A x(t) + B u(t) \\
  y(t) &=& C x(t) + D u(t)
\end{eqnarray*}
has a Hurwitz transfer function.

Reference:  Hassan K. Khalil, {\it Nonlinear Systems}, Prentice Hall, 2002
%%%%%
%%%%%
\end{document}
