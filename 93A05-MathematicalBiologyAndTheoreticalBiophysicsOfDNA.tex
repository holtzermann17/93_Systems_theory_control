\documentclass[12pt]{article}
\usepackage{pmmeta}
\pmcanonicalname{MathematicalBiologyAndTheoreticalBiophysicsOfDNA}
\pmcreated{2013-03-22 18:18:09}
\pmmodified{2013-03-22 18:18:09}
\pmowner{bci1}{20947}
\pmmodifier{bci1}{20947}
\pmtitle{mathematical biology and theoretical biophysics of DNA}
\pmrecord{110}{40921}
\pmprivacy{1}
\pmauthor{bci1}{20947}
\pmtype{Topic}
\pmcomment{trigger rebuild}
\pmclassification{msc}{93A05}
\pmclassification{msc}{92B20}
\pmclassification{msc}{92B05}
\pmclassification{msc}{18-00}
\pmclassification{msc}{18D35}
\pmsynonym{theoretical biology}{MathematicalBiologyAndTheoreticalBiophysicsOfDNA}
\pmsynonym{mathematical biophysics}{MathematicalBiologyAndTheoreticalBiophysicsOfDNA}
%\pmkeywords{what is life}
%\pmkeywords{mathematical principles in biology}
%\pmkeywords{paracrystals}
%\pmkeywords{X-ray diffraction by helices}
%\pmkeywords{Nobel Laureates in Physiology and Medicine}
%\pmkeywords{paracrystal theory}
%\pmkeywords{DNA crystals with partial disorder}
%\pmkeywords{DNA G-quadruplexes in cancers}
%\pmkeywords{in vivo molecular genetics}
%\pmkeywords{st}
\pmrelated{ComplexSystemsBiology}
\pmrelated{CategoryOfSets}
\pmrelated{HamiltonianOperatorOfAQuantumSystem}
\pmrelated{BibliographyForMathematicalBiophysics}
\pmrelated{CategoryOfMRSystems3}
\pmrelated{RobertRosen}
\pmrelated{NicolasRashevsky}
\pmrelated{Quaternions}
\pmrelated{SimilarityAndAnalogousSystemsDynamicAdjointness2}
\pmrelated{PhysicalMathematicsAndEng}
\pmdefines{A- and B- DNA paracrystals}

\endmetadata

\usepackage{amsmath, amssymb, amsfonts, amsthm, amscd, latexsym, enumerate}
\usepackage{xypic, xspace}
\usepackage[mathscr]{eucal}
\usepackage[dvips]{graphicx}
\usepackage[curve]{xy}

\setlength{\textwidth}{6.5in}
%\setlength{\textwidth}{16cm}
\setlength{\textheight}{9.0in}
%\setlength{\textheight}{24cm}

\hoffset=-.75in     %%ps format
%\hoffset=-1.0in     %%hp format
\voffset=-.4in


\theoremstyle{plain}
\newtheorem{lemma}{Lemma}[section]
\newtheorem{proposition}{Proposition}[section]
\newtheorem{theorem}{Theorem}[section]
\newtheorem{corollary}{Corollary}[section]

\theoremstyle{definition}
\newtheorem{definition}{Definition}[section]
\newtheorem{example}{Example}[section]
%\theoremstyle{remark}
\newtheorem{remark}{Remark}[section]
\newtheorem*{notation}{Notation}
\newtheorem*{claim}{Claim}

\renewcommand{\thefootnote}{\ensuremath{\fnsymbol{footnote}}}
\numberwithin{equation}{section}

\newcommand{\Ad}{{\rm Ad}}
\newcommand{\Aut}{{\rm Aut}}
\newcommand{\Cl}{{\rm Cl}}
\newcommand{\Co}{{\rm Co}}
\newcommand{\DES}{{\rm DES}}
\newcommand{\Diff}{{\rm Diff}}
\newcommand{\Dom}{{\rm Dom}}
\newcommand{\Hol}{{\rm Hol}}
\newcommand{\Mon}{{\rm Mon}}
\newcommand{\Hom}{{\rm Hom}}
\newcommand{\Ker}{{\rm Ker}}
\newcommand{\Ind}{{\rm Ind}}
\newcommand{\IM}{{\rm Im}}
\newcommand{\Is}{{\rm Is}}
\newcommand{\ID}{{\rm id}}
\newcommand{\grpL}{{\rm GL}}
\newcommand{\Iso}{{\rm Iso}}
\newcommand{\rO}{{\rm O}}
\newcommand{\Sem}{{\rm Sem}}
\newcommand{\SL}{{\rm Sl}}
\newcommand{\St}{{\rm St}}
\newcommand{\Sym}{{\rm Sym}}
\newcommand{\Symb}{{\rm Symb}}
\newcommand{\SU}{{\rm SU}}
\newcommand{\Tor}{{\rm Tor}}
\newcommand{\U}{{\rm U}}

\newcommand{\A}{\mathcal A}
\newcommand{\Ce}{\mathcal C}
\newcommand{\D}{\mathcal D}
\newcommand{\E}{\mathcal E}
\newcommand{\F}{\mathcal F}
%\newcommand{\grp}{\mathcal G}
\renewcommand{\H}{\mathcal H}
\renewcommand{\cL}{\mathcal L}
\newcommand{\Q}{\mathcal Q}
\newcommand{\R}{\mathcal R}
\newcommand{\cS}{\mathcal S}
\newcommand{\cU}{\mathcal U}
\newcommand{\W}{\mathcal W}

\newcommand{\bA}{\mathbb{A}}
\newcommand{\bB}{\mathbb{B}}
\newcommand{\bC}{\mathbb{C}}
\newcommand{\bD}{\mathbb{D}}
\newcommand{\bE}{\mathbb{E}}
\newcommand{\bF}{\mathbb{F}}
\newcommand{\bG}{\mathbb{G}}
\newcommand{\bK}{\mathbb{K}}
\newcommand{\bM}{\mathbb{M}}
\newcommand{\bN}{\mathbb{N}}
\newcommand{\bO}{\mathbb{O}}
\newcommand{\bP}{\mathbb{P}}
\newcommand{\bR}{\mathbb{R}}
\newcommand{\bV}{\mathbb{V}}
\newcommand{\bZ}{\mathbb{Z}}

\newcommand{\bfE}{\mathbf{E}}
\newcommand{\bfX}{\mathbf{X}}
\newcommand{\bfY}{\mathbf{Y}}
\newcommand{\bfZ}{\mathbf{Z}}

\renewcommand{\O}{\Omega}
\renewcommand{\o}{\omega}
\newcommand{\vp}{\varphi}
\newcommand{\vep}{\varepsilon}

\newcommand{\diag}{{\rm diag}}
\newcommand{\grp}{\mathcal G}
\newcommand{\dgrp}{{\mathsf{D}}}
\newcommand{\desp}{{\mathsf{D}^{\rm{es}}}}
\newcommand{\grpeod}{{\rm Geod}}
%\newcommand{\grpeod}{{\rm geod}}
\newcommand{\hgr}{{\mathsf{H}}}
\newcommand{\mgr}{{\mathsf{M}}}
\newcommand{\ob}{{\rm Ob}}
\newcommand{\obg}{{\rm Ob(\mathsf{G)}}}
\newcommand{\obgp}{{\rm Ob(\mathsf{G}')}}
\newcommand{\obh}{{\rm Ob(\mathsf{H})}}
\newcommand{\Osmooth}{{\Omega^{\infty}(X,*)}}
\newcommand{\grphomotop}{{\rho_2^{\square}}}
\newcommand{\grpcalp}{{\mathsf{G}(\mathcal P)}}

\newcommand{\rf}{{R_{\mathcal F}}}
\newcommand{\grplob}{{\rm glob}}
\newcommand{\loc}{{\rm loc}}
\newcommand{\TOP}{{\rm TOP}}

\newcommand{\wti}{\widetilde}
\newcommand{\what}{\widehat}

\renewcommand{\a}{\alpha}
\newcommand{\be}{\beta}
\newcommand{\grpa}{\grpamma}
%\newcommand{\grpa}{\grpamma}
\newcommand{\de}{\delta}
\newcommand{\del}{\partial}
\newcommand{\ka}{\kappa}
\newcommand{\si}{\sigma}
\newcommand{\ta}{\tau}

\newcommand{\med}{\medbreak}
\newcommand{\medn}{\medbreak \noindent}
\newcommand{\bign}{\bigbreak \noindent}

\newcommand{\lra}{{\longrightarrow}}
\newcommand{\ra}{{\rightarrow}}
\newcommand{\rat}{{\rightarrowtail}}
\newcommand{\ovset}[1]{\overset {#1}{\ra}}
\newcommand{\ovsetl}[1]{\overset {#1}{\lra}}
\newcommand{\hr}{{\hookrightarrow}}

\newcommand{\<}{{\langle}}

%\newcommand{\>}{{\rangle}}
%\usepackage{geometry, amsmath,amssymb,latexsym,enumerate}
%%%\usepackage{xypic}

\def\baselinestretch{1.1}


\hyphenation{prod-ucts}

%\grpeometry{textwidth= 16 cm, textheight=21 cm}

\newcommand{\sqdiagram}[9]{$$ \diagram  #1  \rto^{#2} \dto_{#4}&
#3  \dto^{#5} \\ #6    \rto_{#7}  &  #8   \enddiagram
\eqno{\mbox{#9}}$$ }

\def\C{C^{\ast}}

\newcommand{\labto}[1]{\stackrel{#1}{\longrightarrow}}

%\newenvironment{proof}{\noindent {\bf Proof} }{ \hfill $\Box$
%{\mbox{}}
\newcommand{\midsqn}[1]{\ar@{}[dr]|{#1}}
\newcommand{\quadr}[4]
{\begin{pmatrix} & #1& \\[-1.1ex] #2 & & #3\\[-1.1ex]& #4&
 \end{pmatrix}}
\def\D{\mathsf{D}}

\begin{document}
\section{Introduction}

\emph{Mathematical biology} (also known as \emph{theoretical biology}) is the study of biological
principles and laws, together with the formulation of mathematical models-- and also the logical and mathematical representation \cite{NR65}-- of \emph{complex biological systems} at all levels of biological organization, from the quantum/molecular level to the physiological, systemic and the whole organism levels.

\section{Mathematical biophysics}

\subsection{History}
Mathematical biophysics has dominated for over half a century developments in mathematical biology
as theoretical or mathematical physicists have expanded their interests to applying mathematical and
physical concepts to studying living organisms and in repeated attempts to `define life itself' \cite{ES45,RR97}.

A prominent early example was the famous Erwin $Schr\"odinger$ 's book (published in 1945 in Cambridge, UK)
entitled suggestively ``What is Life?", and that was perhaps too critically re-evaluated a decade ago by Robert Rosen.
This interesting and concise book appears to have inspired a decade later the discovery of the
double helical, molecular structures of A- and B- DNA crystals/paracrystals (\cite{RH-SNB62,ICB74,ICB80,ICBetal2k9}) by Maurice Wilkins, Rosalind Franklin, Francis Crick and James D. Watson, with the first two (bio)physical chemists working at that time with X-ray Diffraction of DNA crystals at
\PMlinkexternal{King's College in London}{http://www.kcl.ac.uk/schools/biohealth/graduate/taught/molbiophysics/}, (see also the websites about
\PMlinkexternal{Rosalind Franklin and Maurice Wilkins}{http://www.kcl.ac.uk/college/history/people/franklin_wilkins.html}), and the last two researchers working at
\PMlinkexternal{The Cavendish Laboratory of the University of Cambridge(UK)}{http://www.phy.cam.ac.uk/}, (see also
\PMlinkexternal{related news at}{http://www.admin.cam.ac.uk/news/dp/2004072903}, and also the
\PMlinkexternal{new Biology and Physics of Medicine Laboratory}{http://www.pom.cam.ac.uk/initiative.html} at
\PMlinkexternal{The Cavendish}{http://www.pom.cam.ac.uk/initiative.html}). With the notable exception of Rosalind Franklin and Robert Rosen, the other three mathematical and experimental biophysicists became
\PMlinkexternal{Nobel Laureates in Physiology and Medicine}{http://nobelprize.org/nobel_prizes/medicine/laureates/1962/}.

Notably DNA configurations \emph{in vivo} include a significant amount of dynamic, partial disorder and may be defined at best as \emph{paracrystals} (\cite{RH-SNB62,ICB74,ICB80}), a fact which has important consequences for functional biology and \emph{in vivo} molecular genetics. Moreover, other structures (such as Z-DNA) were discovered in certain organisms, and other configurations were found under physiological conditions (see, for example, the excellent, DNA structure representations rendered by computers on pp. 852-854 in Voet and Voet, 1995 \cite{VoetD-JG95}, as well as a recent update review \cite{ICBetal2k9} and earlier generalizations \cite{ICB87a,ICB87b}), such as the \PMlinkexternal{DNA G-quadruplexes that can control gene transcription and translation
- especially in cancers}{http://www.phy.cam.ac.uk/research/bss/molbiophysics.php}.

Erwin $Schr\"odinger$'s fundamental contribution to Quantum Mechanics preceded the others discussed in the previous paragraph by more than two decades when he formulated the fundamental equations of Quantum Mechanics which bear his name, and modestly called the operator appearing in the $Schr\"odinger$ equations the
\PMlinkname{``Hamiltonian operator"}{HamiltonianOperatorOfAQuantumSystem} - a term universally employed in the Theoretical and Mathematical Physics literature that now bears the name of the distinguished Irish physicist, Sir William Rowan Hamilton. Hamilton is now also considered to be one of the world's greatest mathematicians (see for example, his introduction of the concept of \emph{quaternions} in 1835), and he was also the first foreign Member to be elected to the US National Academy of Sciences in 1865. Subsequently, Schr\"odinger was awarded a Nobel Prize for his fundamental, theoretical (and mathematical) physics contribution by the Stockholm Nobel Committee, and soon thereafter in 1941 became the Director of the (Dublin) Institute for Advanced Studies (DIAS) in Ireland, instead of joining Albert Einstein on the staff at Princeton's Institute for Advanced Studies.

\section{Recent developments}


Robert Rosen (1937-1998) was a prominent relational biologist who completed his PhD studies with
Nicolas Rashevsky, the former Head of the Committee for Mathematical Biology at the
University of Chicago, USA, with a Thesis on Relational Biology (Metabolic-Replication Systems, or
$(M,R)$-systems). His publications (see bibliography) include an impressive number of volumes and textbooks on Theoretical Biology, Relational Biology, Anticipation, Ageing, Complex Dynamical Systems in Biology, (Bio) Chemical Morphogenesis and Quantum Genetics. He also reported in 1958 the first abstract representation of living organisms in special, small \emph{categories of sets} called \emph{categories of metabolic--replication systems}, or
\PMlinkname{category of $(M,R)$-systems}{CategoryOfMRSystems3}.

To quote Robert Rosen:

\begin{quote}
\emph{``Ironically, the idea that life requires an explanation is a relatively new one. To the ancients life simply \emph {was}; it was a given; a first principle..."}
\end{quote}

One might add also that to most biologists ``Life" is still a given, but something that might be `explained by reduction
to genes, nucleic acids, enzymes and small biomolecules', i.e. some sort of ordered `bag' of biochemicals mostly filled
with aqueous solutions inside selective biomembranes, etc. Robert Rosen's viewpoint was quite different from this:
he saw life as a \emph{dynamic, relational pattern} in categories of metabolic-repair (open) systems characterized
by flows--relational/material, energetic and informational processes-- perhaps closer to the injunction by Heraclitus of \emph {``panta rhei"-everything flows}, but with the very important addition that life flows in a \emph {uniquely complex relational pattern} that is observed only in living systems, thus perhaps uniquely defining \emph{Life} as a special, \emph{super-complex process} (\cite{BBGG06}. Once life stops-- even though the material structure is still there-- the essential relational flow (related to energetic, informational as well as material) patterns are gone forever, with the possible exceptions of the `raising from the dead in the Egyptian myths about Osiris' , and also in certain well-known sections of the New Testament.



\begin{thebibliography}{9}

\bibitem{ES45}
Erwin Schroedinger.1945. \emph{What is Life?}. Cambridge University Press: Cambridge (UK).

\bibitem{NR54}
Nicolas Rashevsky.1954, Topology and life: In search of general mathematical principles in
biology and sociology, \emph{Bull. Math. Biophys.} 16: 317-348.

\bibitem{NR65}
Nicolas Rashevsky. 1965. Models and Mathematical Principles in Biology. In: Waterman/Morowitz, \emph{Theoretical and Mathematical Biology}, pp. 36-53.

\bibitem{REF53}
Rosalind E. Franklin and R.G. Gosling. 1953. Evidence for 2-chain helix in crystalline structure
of sodium deoxyribonucleate (DNA). \emph{Nature} 177: 928-930.

\bibitem{WM-SW-SR-HRW53}
Wilkins, M.H.F. et al. 1953. Helical structure of crystalline deoxypentose nucleic acid (DNA).
\emph{Nature} 172: 759-762.

\bibitem{FHCC53}
Francis H.C. Crick. 1953. Fourier transform of a coiled coil. \emph{Acta Cryst}. 6: 685-687

\bibitem{HRW68}
H. R. Wilson. 1966. \emph{Diffraction of X-rays by Proteins, Nucleic Acids and Viruses}.
London: Arnold.

\bibitem{BBGG06}
I. C. Baianu, J. F. Glazebrook, R. Brown and G. Georgescu.: Complex Nonlinear Biodynamics in Categories, Higher dimensional Algebra and \L{}ukasiewicz-Moisil Topos: Transformation of Neural, Genetic and Neoplastic Networks, \emph{Axiomathes}, \textbf{16}: 65-122(2006).
\PMlinkexternal{available here as PDF}{http://www.bangor.ac.uk/~mas010/pdffiles/Axio7complx_Printedk7_v17p223_fulltext.pdf}

\bibitem{ICB74}
I.C. Baianu. 1974. Ch.4 in \emph{Structural Studies by X-ray Diffraction and Electron Microscopy of Erythrocite and Bacterial Plasma Membranes}. PhD Thesis, London: a University of London Library publication.

\bibitem{ICB6}
Baianu, I.C.: 1977, A Logical Model of Genetic Activities in \L ukasiewicz Algebras: The Non-linear Theory. \emph{Bulletin of Mathematical Biology}, \textbf{39}: 249-258.

\bibitem{ICB78}
I.C. Baianu. 1978. X-ray Scattering by Partially Disordered Membrane Lattices.
\emph{Acta Crystall}. \textbf{A34}: 731-753. (\emph{paper contributed from The Cavendish Laboratory, Cambridge in 1979}).

\bibitem{ICB80}
I.C. Baianu. 1980. Structural Order and Partial Disorder in Biological Systems. \emph{Bull. Math. Biol.}, 42: 186-191.
(\emph{paper contributed from The Cavendish Laboratory, Cambridge in 1979}).

\bibitem{IBCGHpnas84}
I. C. Baianu, C. Critchley, Govindjee, AND H. S. Gutowsky. 1984.
NMR study of chloride ion interactions with thylakoid membranes
{\em Proced. Natl. Acad. Sci. USA (Biophysics)}., Vol. 81, pp. 3713--3717, June 1984,
(Key words: photosynthesis/oxygen evolution/chloride binding/chlorine--35/halophytes).


\bibitem{ICB87a}
Baianu, I. C.: 1986-1987a, Computer Models and Automata Theory in Biology and Medicine., in M. Witten (ed.), \emph{Mathematical Models in Medicine}, vol. 7., Ch.11 Pergamon Press, New York, 1513 -1577;
available downloads as: \emph{CERN Preprint No. EXT-2004-072}-
\PMlinkexternal{CERN Preprint as PDF}{http://doe.cern.ch//archive/electronic/other/ext/ext-2004-072.pdf}, or
\PMlinkexternal{as external html document}{http://en.scientificcommons.org/1857371} .

\bibitem{ICB87b}
Baianu, I. C.: 1987b, Molecular Models of Genetic and Organismic Structures, in \emph{Proceed. Relational Biology Symp.}
Argentina; \PMlinkexternal{CERN Preprint No.EXT-2004-067}{http://doc.cern.ch/archive/electronic/other/ext/extusers/2004
67/MolecularModels-ICB3.doc}.

\bibitem{ICB2}
Baianu, I. C.: 1983, Natural Transformation Models in Molecular Biology., in \emph{Proceedings of the SIAM Natl. Meet}., Denver,CO.; \PMlinkexternal{Eprint:}{\\http://cogprints.org/3675/} and
\PMlinkexternal{html document}{http://cogprints.org/3675/0l/Naturaltransfmolbionu6.pdf}.

\bibitem{ICB2}
Baianu, I.C.: 1984, A Molecular-Set-Variable Model of Structural and Regulatory Activities in Metabolic and Genetic Networks, \emph{FASEB Proceedings} \textbf{43}, 917.

\bibitem{ICB2004a}
Baianu, I.C.: 2004a. \L{}ukasiewicz-Topos Models of Neural Networks, Cell Genome and Interactome Nonlinear Dynamic Models (2004). Eprint. Cogprints--Sussex Univ.

\bibitem{ICB04b}
Baianu, I.C.: 2004b \L{}ukasiewicz-Topos Models of Neural Networks, Cell Genome and Interactome Nonlinear Dynamics). CERN Preprint EXT-2004-059. \textit{Health Physics and Radiation Effects} (June 29, 2004).

\bibitem{BGG2k4}
Baianu, I. C., Glazebrook, J. F. and G. Georgescu: 2004, Categories of Quantum Automata and N-Valued \L ukasiewicz Algebras in Relation to Dynamic Bionetworks, \textbf{(M,R)}--Systems and Their Higher Dimensional Algebra, \PMlinkexternal{Abstract and Preprint of Report as PDF}{\\http://www.ag.uiuc.edu/fs401/QAuto.pdf} or
as \PMlinkexternal{an html document}{http://www.medicalupapers.com/quantum+automata+math+categories+baianu/}.

\bibitem{ICB9}
Baianu, I. C.: 2004b, Quantum Interactomics and Cancer Mechanisms,
\PMlinkexternal{Preprint No. 00001978}{http: bioline.utsc.utoronto.ca/archive/00001978/01/\\
Quantum Interactomics In Cancer--Sept13k4E-- cuteprt.pdf}.

\bibitem{ICB2k6}
Baianu, I. C.: 2006, Robert Rosen's Work and Complex Systems Biology, \emph{Axiomathes} \textbf{16}(1--2):25--34.

\bibitem{BBGGk6}
Baianu I. C., Brown R., Georgescu G. and J. F. Glazebrook: 2006, Complex Nonlinear Biodynamics in Categories, Higher Dimensional Algebra and \L ukasiewicz-Moisil Topos: Transformations of Neuronal, Genetic and Neoplastic Networks., \emph{Axiomathes}, \textbf{16} Nos. 1-2: 65-122.

\bibitem{BBGk7}
Baianu, I.C., R. Brown and J.F. Glazebrook. : 2007, Categorical Ontology of Complex Spacetime Structures: The Emergence of Life and Human Consciousness, \emph{Axiomathes}, \textbf{17}: 35-168.

\bibitem{ICBetal2k9}
Baianu, I.C. et al. 2009. \PMlinkexternal{``DNA Molecular Structure, Dynamics and Spectroscopy.'' (Free GNUL download.)}{http://planetphysics.org/?op=getobj&from=books&id=220}

\bibitem{RH-SNB62}
R. Hosemann and S. N. Bagchi. 1962. \emph{Direct Analysis of Diffraction by Matter}. Amsterdam: North Holland.

\bibitem{VoetD-JG95}
D. Voet and J.G. Voet. 1995. \emph{Biochemistry}. 2nd Edition, New York, Chichester, Brisbone, Toronto,
Singapore: J. Wiley and Sons, INC., 1,361 pages, over 3,000 high-resolution molecular models in color -- (\emph{an excellently illustrated textbook})

\bibitem{RR97}
Robert Rosen. 1997 and 2002. \emph{Essays on Life Itself}.

\bibitem{RRosen1}
Rosen, R.: 1958a, A Relational Theory of Biological Systems
\emph{Bulletin of Mathematical Biophysics} \textbf{20}: 245-260.

\bibitem{RRosen2}
Rosen, R.: 1958b, The Representation of Biological Systems from the Standpoint of the Theory of Categories.,
\emph{Bulletin of Mathematical Biophysics} \textbf{20}: 317-341.

\bibitem{RRosen60}
Rosen, R. 1960. A quantum-theoretic approach to genetic problems. \emph{Bulletin of Mathematical Biophysics}
22: 227-255.

\bibitem{RRosen87}
Rosen, R.: 1987, On Complex Systems, \emph{European Journal of Operational Research}
\textbf{30}, 129-134.

\bibitem{RR70}
Rosen,R. 1970, \emph{Dynamical Systems Theory in Biology}. New York: Wiley Interscience.

\bibitem{RR70}
Rosen,R. 1970, \emph{Optimality Principles in Biology}, New York and London: Academic Press.

\bibitem{RR70}
Rosen,R. 1978, \emph{Fundamentals of Measurement and Representation of Natural Systems}, Elsevier Science Ltd,

\bibitem{RR70}
Rosen,R. 1985, \emph{Anticipatory Systems: Philosophical, Mathematical and Methodological Foundations}. Pergamon Press.

\bibitem{RR91}
Rosen,R. 1991, \emph{Life Itself: A Comprehensive Inquiry into the Nature, Origin, and Fabrication of Life}, Columbia University Press

\bibitem{EC84}
Ehresmann, C.: 1984, \emph{Oeuvres compl\`etes et comment\'ees: Amiens, 1980-84}, edited and commented
by Andr\'ee Ehresmann.

\bibitem{EACV2}
Ehresmann, A. C. and J.-P. Vanbremersch: 2006, The Memory Evolutive Systems as a Model of Rosen's Organisms,
in \emph{Complex Systems Biology}, I.C. Baianu, Editor, \emph{Axiomathes} \textbf{16} (1--2), pp. 13-50.

\bibitem{EML1}
Eilenberg, S. and Mac Lane, S.: 1942, Natural Isomorphisms in Group Theory., \emph{American Mathematical Society 43}: 757-831.

\bibitem{EL}
Eilenberg, S. and Mac Lane, S.: 1945, The General Theory of Natural Equivalences,
\emph{Transactions of the American Mathematical Society} 58: 231-294.

\bibitem{Elsasser}
Elsasser, M.W.: 1981, A Form of Logic Suited for Biology., In: Robert, Rosen, ed., \emph{Progress in Theoretical Biology}, Volume 6, Academic Press, New York and London, pp 23-62.

\end{thebibliography}

%%%%%
%%%%%
\end{document}
